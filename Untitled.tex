\documentclass[english,man]{apa6}

\usepackage{amssymb,amsmath}
\usepackage{ifxetex,ifluatex}
\usepackage{fixltx2e} % provides \textsubscript
\ifnum 0\ifxetex 1\fi\ifluatex 1\fi=0 % if pdftex
  \usepackage[T1]{fontenc}
  \usepackage[utf8]{inputenc}
\else % if luatex or xelatex
  \ifxetex
    \usepackage{mathspec}
    \usepackage{xltxtra,xunicode}
  \else
    \usepackage{fontspec}
  \fi
  \defaultfontfeatures{Mapping=tex-text,Scale=MatchLowercase}
  \newcommand{\euro}{€}
\fi
% use upquote if available, for straight quotes in verbatim environments
\IfFileExists{upquote.sty}{\usepackage{upquote}}{}
% use microtype if available
\IfFileExists{microtype.sty}{\usepackage{microtype}}{}

% Table formatting
\usepackage{longtable, booktabs}
\usepackage{lscape}
% \usepackage[counterclockwise]{rotating}   % Landscape page setup for large tables
\usepackage{multirow}		% Table styling
\usepackage{tabularx}		% Control Column width
\usepackage[flushleft]{threeparttable}	% Allows for three part tables with a specified notes section
\usepackage{threeparttablex}            % Lets threeparttable work with longtable

% Create new environments so endfloat can handle them
% \newenvironment{ltable}
%   {\begin{landscape}\begin{center}\begin{threeparttable}}
%   {\end{threeparttable}\end{center}\end{landscape}}

\newenvironment{lltable}
  {\begin{landscape}\begin{center}\begin{ThreePartTable}}
  {\end{ThreePartTable}\end{center}\end{landscape}}

  \usepackage{ifthen} % Only add declarations when endfloat package is loaded
  \ifthenelse{\equal{\string man}{\string man}}{%
   \DeclareDelayedFloatFlavor{ThreePartTable}{table} % Make endfloat play with longtable
   % \DeclareDelayedFloatFlavor{ltable}{table} % Make endfloat play with lscape
   \DeclareDelayedFloatFlavor{lltable}{table} % Make endfloat play with lscape & longtable
  }{}%



% The following enables adjusting longtable caption width to table width
% Solution found at http://golatex.de/longtable-mit-caption-so-breit-wie-die-tabelle-t15767.html
\makeatletter
\newcommand\LastLTentrywidth{1em}
\newlength\longtablewidth
\setlength{\longtablewidth}{1in}
\newcommand\getlongtablewidth{%
 \begingroup
  \ifcsname LT@\roman{LT@tables}\endcsname
  \global\longtablewidth=0pt
  \renewcommand\LT@entry[2]{\global\advance\longtablewidth by ##2\relax\gdef\LastLTentrywidth{##2}}%
  \@nameuse{LT@\roman{LT@tables}}%
  \fi
\endgroup}


\ifxetex
  \usepackage[setpagesize=false, % page size defined by xetex
              unicode=false, % unicode breaks when used with xetex
              xetex]{hyperref}
\else
  \usepackage[unicode=true]{hyperref}
\fi
\hypersetup{breaklinks=true,
            pdfauthor={},
            pdftitle={Socio-economic status and word comprehension of infants: a meta-analytic review},
            colorlinks=true,
            citecolor=blue,
            urlcolor=blue,
            linkcolor=black,
            pdfborder={0 0 0}}
\urlstyle{same}  % don't use monospace font for urls

\setlength{\parindent}{0pt}
%\setlength{\parskip}{0pt plus 0pt minus 0pt}

\setlength{\emergencystretch}{3em}  % prevent overfull lines

\ifxetex
  \usepackage{polyglossia}
  \setmainlanguage{}
\else
  \usepackage[english]{babel}
\fi

% Manuscript styling
\captionsetup{font=singlespacing,justification=justified}
\usepackage{csquotes}
\usepackage{upgreek}

 % Line numbering
  \usepackage{lineno}
  \linenumbers


\usepackage{tikz} % Variable definition to generate author note

% fix for \tightlist problem in pandoc 1.14
\providecommand{\tightlist}{%
  \setlength{\itemsep}{0pt}\setlength{\parskip}{0pt}}

% Essential manuscript parts
  \title{Socio-economic status and word comprehension of infants: a meta-analytic
review}



  \author{Camila Scaff\textsuperscript{1}~\&}

  \def\affdep{{"", ""}}%
  \def\affcity{{"", ""}}%


  \authornote{
    \newcounter{author}
    Complete departmental affiliations for each author (note the
    indentation, if you start a new paragraph). Enter author note here.

                      Correspondence concerning this article should be addressed to Camila Scaff, FILL IN.
                          }


  \keywords{keywords \\

    \indent Word count: X
  }





\usepackage{amsthm}
\newtheorem{theorem}{Theorem}
\newtheorem{lemma}{Lemma}
\theoremstyle{definition}
\newtheorem{definition}{Definition}
\newtheorem{corollary}{Corollary}
\newtheorem{proposition}{Proposition}
\theoremstyle{definition}
\newtheorem{example}{Example}
\theoremstyle{definition}
\newtheorem{exercise}{Exercise}
\theoremstyle{remark}
\newtheorem*{remark}{Remark}
\newtheorem*{solution}{Solution}
\begin{document}

\maketitle

\setcounter{secnumdepth}{0}



\subsection{Intro}\label{intro}

It has been stated during decades now the importance of early
development for later cognitive skills (REF). A vast number of public
policies in various countries has been applied to ameliorate and promote
better \enquote{early days} for infants. Among the different policies
and strategies, a special focus has been given to language development
and achievements during infancy. Many interventions assessing cognitive
skills in general actually focus mainly or only in language acquisition
and unfolding (many refs haha). This seems a coherent move from
governments and interventions because also studies have shown the
correlation between early language skills and cognitive skills, a later
to school achievement. Better educated parents have been found to have
young children with more advanced language development (Burchinal,
Peisner-Feinberg, Pianta, \& Howes, 2002; Gest, Freeman, Domitrovich, \&
Welsh, 2004; Hoff \& Tian, 2005; Raviv, Kessenich, \& Morrison, 2004;
Turner \& Johnson, 2003) and higher overall IQ (Linver et al., 2002).
Interestingly, more educated mothers seem to expect their children to
say first sounds and words and to \enquote{think} sooner (Hoff et al.,
2002). Expecting more advanced child vocabulary will likely increase
parents efforts to encourage younger children's learning experiences via
clear and responsive communication (see also Lareau, 2003). In another
body of literature it has been shown that low socio-economic status
presents a disadvantage in various aspects of life, and from very early
on. Differences have been shown in several cognitive aspects during
adullthood \& childhood. Also various studies have point-out differences
in various linguistics levels, starting from early childhood and
infancy, specifically vocabulary. We aim to review the effect of SES in
early lexical acquisition, focusing only in infancy : up to 3yo. They
are available already metanalytic reviews but not for the first period
of life. ref for children and review for adolescents.We selected
vocabulary as the outcome because of its critical importance to
development across childhood, adolescence, and adulthood (Schoon,
Parsons, Rush, \& Law, 2010). Of particular relevance, children's
ability to understand a variety of words is an essential component of
kindergarten readiness (Bierman et al., 2008; Doherty, 1997; Whitehurst
\& Lonigan, 1998). For instance, successful entry into kindergarten
requires basic skills, many dependent on vocabulary, including abilities
to understand explanations and follow instructions. Children with
limited vocabularies should experience more difficulty during classroom
activities. Such early academic problems, rather than fading with time,
may place students on a persistent trajectory of academic problems
(Shonkoff \& Philips, 2000), including repeating a grade, requiring
special education services, or leaving high school without obtaining a
diploma (Brooks-Gunn, Guo, \& Furstenberg, 1993; Ramey \& Ramey, 2004).
Moreover, young children with more advanced vocabulary do better in
school over time and demonstrate greater academic achievement (Jorgenson
\& Jorgenson, 1996). Consequently, preschool-aged children with larger
vocabularies will likely eventually achieve more academically on average
and thus have access to more diverse higher educational and career
opportunities. Similarly, it is likely that the young adults who
achieved the most academically typically had above average vocabulary
skills as young children.

\subsection{Measuring SES}\label{measuring-ses}

Socioeconomic status refers to specific quote. But in general reflects
the situation of the household in terms of economic and social resources
Different ways to estimate this measure that ranges from single measures
such as annual income or maternal education, or composites taking into
account both of these criteria plus parental occupation or neighborhood.
Research has suggested to look into these different measures as
independent factors Others argue that Maternal education, the most
common in children language acquistion correlates well enough with a
often use composite measure; the Hollingshed index ref + definition For
this study we take into account the all range of SES presented in the
litterature. We ackowledge that dichotomies such as High and Low SES are
arbitrary decisions taken at a given time for a given population Just as
farah et al executive function paper with profit from the variation of
the measure and apply it as moderator in order to examine whether the
measures used to estimate child socioeconomic studies influence the
strength of the SES - word comprehension relationship.

\subsection{Measuring word
comprehension}\label{measuring-word-comprehension}

\subsection{Goals of the current
meta-analysis}\label{goals-of-the-current-meta-analysis}

Method

Search procedures and selection of studies Studies were identified
through searches of the \enquote{Pubmed} engine throughout September
2016 using as keywords \enquote{Socio-economic-status}
\enquote{language} filter from birth to 23 months, and exclusively using
the same words but filtering only for preschool. This search was on
purpose not focusing specifically in vocabulary but in language in
general, by this mean a big database was created with The search
required that at least it was mentioned in the abstract studies
evaluating language in the appropriate age range. Identified studies
required

1.1 Literature search

1.2 Inclusion criteria

1.3 Eligible word comprehension and SES measures

1.4 selection of studies

\begin{enumerate}
\def\labelenumi{\arabic{enumi}.}
\setcounter{enumi}{1}
\tightlist
\item
  Effect size and moderator coding procedure
\end{enumerate}

2.1 Effect size coding

2.2 Moderator Coding

Sample characteristics

Age range Intended sampleSES Amount of SES variability in the sample
Extent of exclusionary criteria Racial/ethnic composition Mean age Sex
composition

Effect size characteirstics

Category of SES construct Number of measures used to calculate the SES
variable Category of Word comprehension construct Number of measures
used to calculate the word comprehension variable

Publication characteristics

Type of publication SES as a primary focus

\begin{enumerate}
\def\labelenumi{\arabic{enumi}.}
\setcounter{enumi}{2}
\tightlist
\item
  Analytical procedures
\end{enumerate}

3.1 Calculating average effect size 3.2 statistical independence 3.3
Fixed and random effect models 3.4 Test for heterogeneity 3.5 Moderator
Analysis 3.5 Test for publication bias

RESULTS

Study characteristics Overall effect size Tests for heterogeneity

\begin{enumerate}
\def\labelenumi{\arabic{enumi}.}
\setcounter{enumi}{3}
\tightlist
\item
  Moderator analysis 4.1 Sample chracteristics 4.2 Effect size
  charateristics 4.3 Publication characteristics 4.4 publication bias
\end{enumerate}

Discussion

Eligibility Criteria for Study Inclusion Typing \enquote{language} \&
SES in pubmed search

We excluded studies that reported vocabulary outcomes and socioeconomic
status differences. as well as studies that only reported outcomes in
language or executive function. If two or more studies referred to the
same population, the study that had evaluated the effect of SES on
cognitive outcome as a stated objective was included in the review. If
two or more, or none, of the studies referring to the same population
had that objective, the most recently published article that reported
the cognitive outcome of the partici- pants at an older age was
selected.

Data Sources and Search Strategy

We searched electronic databases MEDLINE, EMBASE, PsycINFO and Social
Science Citation Index using the following search terms, both as
keywords and as subject headings: ((\enquote{`preterm}' or
\enquote{`premature}') and (\enquote{`birth}' or \enquote{`delivery}' or
\enquote{`infant}')) or (\enquote{`prematurity}' or \enquote{`low birth
weight}') and (\enquote{`social}' or \enquote{`socioeconomic}' or
\enquote{`sociode- mographic}' or \enquote{`environment\emph{'') and
(``intelligence'' or ``IQ'' or ``cogniti}}' or \enquote{`academic}' or
\enquote{`development}'). The \enquote{`explode}' feature was used with
subject headings to include articles categorized under more specific
subhead- ings. The search was restricted to English-language articles
published between January 1990 and July 2011, in order to avoid study
populations born before 1990. The title and abstract of all studies
retrieved from the electronic search were screened to identify
case--control or cohort studies that reported cognitive outcomes among
children born preterm, VLBW or ELBW. The full text of all relevant
studies identified from the initial screen were then evaluated to select
articles that reported the effect of at least one SES indicator on
cognitive outcome. The elec- tronic search was supplemented by a manual
search of the reference lists of studies that met the eligibility
criteria for inclusion.

Study Quality Assessment We assessed the quality of included studies
based on their design, representativeness of the study population,
quality of the SES data, quality of the outcome data and appro-
priateness of statistical analyses, using an appraisal checklist adapted
from the Quality Assessment Tool for Quantitative Studies of the
Effective Public Health Practice Project {[}20{]} (Table 1). The aim of
the appraisal is to pro- vide a descriptive score of the external
validity of the effect of SES reported by the studies. Each study was
given a component rating of \enquote{`strong}', \enquote{`moderate}' or
\enquote{`weak}' in each of the five areas assessed. A global quality
rating was then assigned to the studies based on the component rat-
ings. Studies that did not receive any \enquote{`weak}' component rating
were judged \enquote{`strong}' globally. Studies that received one
\enquote{`weak}' component rating were assigned a global rating of
\enquote{`moderate}' and studies were considered \enquote{`weak}'
globally if they received two or more \enquote{`weak}' component
ratings. No study was excluded from the review on the basis of its
quality

Data Extraction

From each included study, we extracted data on: (1) study
characteristics, such as population, setting and sampling methods; (2)
cognitive measures including participants' ages at cognitive assessment
and type of assessment tools employed; (3) types of SES indicators used
and the defi- nitions for categorization; and (4) statistical tests used
and the direction, magnitude and statistical significance of the effect
of each SES indicator on cognitive outcome. The variables included as
confounders in multivariable analyses were also recorded. If the article
reported results from repeated cognitive assessment at different time
points, the data obtained from participants at the older age were
extracted. No attempt was made to contact authors for additional data
missing from the published article.

Data Synthesis

SES indicators were classified into four categories: indi- vidual-level,
family-structure, contextual and composite indicators. The number of
studies that evaluated each SES indicator and the proportion of studies
that reported a sta- tistically significant effect of the SES indicator
on cogni- tive outcome were calculated. The range of the magnitude of
effect of SES indicators on cognitive outcomes reported by the studies
was recorded, focusing on results that had been appropriately adjusted
for confounders. The 5 \%level was used to define statistical
significance

Flowchart The literature search yielded 4,162 unique articles with 19
studies meeting the eligibility criteria. Seven studies were based on
the same three populations. As none of these studies had specifically
aimed to evaluate the effect of SES on cognitive outcome, the most
recently published article was selected, resulting in 15 studies
{[}21--35{]} included in the review (Fig. 1). All included studies had
adopted a longi- tudinal cohort design. No additional study was
identified from the manual search. The characteristics of the studies
are summarized in Table 2. Thirteen studies reported the effect of SES
on cognitive outcome assessed at a single point in time {[}21--33{]}. Of
these, five had conducted the assessment before the age of 2 years
{[}26, 28, 29, 32, 33{]}, four at pre- school-age (ages 3--4 years)
{[}23--25, 30{]} and four at school- age (age older than 5 years) {[}21,
22, 27, 31{]}. Three studies reported the effect of SES on the change in
cognitive status over time {[}32, 34, 35{]}. Depending Depending on the
cognitive assessment employed, cognitive scores were expressed as
developmental quo- tients, Mental Development Index (MDI) from the
Bayley Scales of Development, Mental Processing Composite from the
Kaufman Assessment Battery for Children, reading and mathematics scores
from the Wechsler Indi- vidual Achievement Test-II and IQ. Overall, the
quality of the data on SES was high (Table 3). Three studies {[}22, 25,
30{]} received \enquote{`weak} Matern Child Health J (2013)
17:1689--1700 component ratings for their analyses as they did not ade-
quately adjust for confounders and had reported the effect of SES on
cognitive outcome using bivariate analyses. Types of SES Indicator
Thirteen indicators of SES were evaluated by the included studies (Table
4). Maternal educational level was the most frequently used SES
indicator. Maternal age at birth and race/ethnicity were used as proxy
indicators of SES. The use of medical insurance status as a SES
indicator was unique to studies based in the USA. The composite indi-
cator of \enquote{`social risk}' comprised several individual-level
indicators. Piecuch et al. {[}30{]} defined high \enquote{`social risk}'
as one or more of maternal education under 12 years, com- plete
unemployment in a household and dependence on government assistance for
health insurance. Hack et al. {[}26{]} derived an ordinal
\enquote{`social risk}' score based on maternal marital status, race and
educational level. In a separate study, Hack et al. {[}34{]} derived a
composite indi- cator of SES using the mean of z scores of maternal edu-
cation and neighbourhood median family income. Both contextual
indicators listed in Table 4 were used in the study by Hack et al.
{[}34{]} to describe the effect of SES on the the change in cognitive
status over time.

Nous avons effectué une méta-analyse sur un corpus de 12 articles
portant sur l'impact du niveau socio-économique des parents sur le
vocabulaire de leur enfant. Pour sélectionner ces articles nous avons
choisi des critères en rapport avec notre étude expérimentale
précédente. Nous nous sommes concentrés sur les mêmes types de tests que
nous avons réalisé, c'est à dire le CDI (IFDC en français), le CCT mais
aussi le Peabody Picture Vocabulary Test (PPVT) car c'est une version
non numérique du CCT où l'enfant doit montrer parmi 2 ou 4 images celle
nommée par l'expérimentateur (Rice \& Watkins, 1996). Nous avons donc
étudié la compréhension (CCT, PPVT, CDI) et la production (CDI) du
vocabulaire des enfants. Les enfants devaient aussi être âgés de moins
de 42 mois. J'ai sélectionné ce sous-ensemble d`articles d'une plus
grande sélection réalisée par l'équipe. Cette grande sélection résulte
d'une recherche sur pubmed avec des critères plus larges : rassembler
tous les articles qui traitent de l'influence du niveau socio-économique
sur le langage de l'enfant de moins de 6 ans. Du fait de mes critères
précis cités plus haut, je n'ai pas pu inclure un grand nombre
d'articles, c'est pourquoi ma méta-analyse porte sur 12 articles au
total. L'objectif est d'estimer la taille d'effet, une mesure qui
indique la force de la relation entre deux variables. Ici nous cherchons
à observer la force de la relation entre le milieu socio-économique et
le vocabulaire de l'enfant. Pour intégrer les résultats dans une analyse
quantitative, il est nécessaire d'avoir des échantillons mutuellement
indépendants et non liés. Si ces échantillons sont liés, c'est-à-dire si
plusieurs mesures sont effectuées sur les mêmes enfants à l'intérieur
d'un papier, nous les fusionnons en prenant la moyenne. Par exemple, si
à l'intérieur d'un papier, les mêmes enfants sont testés avec plusieurs
tests de compréhension (CDI Compréhension + CCT), alors ces échantillons
sont considérés comme liés, et donc leurs corrélations sont fusionnées.
Pour les données où nous n'avons pas une corrélation mais plutôt un
contraste entre deux groupes, nous utilisons les différences de moyenne
standardisée d, puis appliquons une formule de conversion d → r. Pour
déterminer la précision de chaque taille d'effet, nous calculons
l'erreur type qui prend en compte le nombre d'enfants testés. Plus ce
nombre d'enfants testés est grand, plus l'étude est précise, et donc
plus elle a du poids dans l'analyse globale. Afin que les corrélations
soient comparables, nous utilisons la transformation de Fisher pour les
convertir en z.

Results Les analyses ont été réalisées principalement sur excel, puis
vérifiés avec des packages R (metafor, Viechtbauer 2010). Pour pouvoir
analyser toutes les données de la littérature et pouvoir les comparer
entre elles, nous réalisons un graphique en entonnoir, qui montre la
taille d'effet en fonction de la précision de l'étude. Plus le nombre de
sujet d'une étude est grand, plus l'étude est considérée comme précise
et donc plus elle se rapproche de la valeur moyenne de l'ensemble des
études, ce qui représente notre meilleure estimation de la vraie valeur
d'association. Ces études où la précision est importante se trouvent
donc en haut du graphique, proche de la moyenne puisque l'intervalle de
confiance, et donc les fluctuations d'échantillonnage sont moins
importants. Cependant, nous remarquons que les données les plus précises
ne sont pas toujours celles qui se rapprochent de la moyenne et donc de
la vraie valeur. Nous observons que l'étude la plus précise des données
étudiant la compréhension (figure 7) correspond à celle où les enfants
sont les plus jeunes (figure 8). C'est donc peut être dû au fait que les
enfants sont très jeunes comparés aux autres enfants, que cette taille
de l'effet s'éloigne de la valeur moyenne. Un graphique en entonnoir
permet aussi de visualiser s'il existe des biais de publication. Pour
qu'il n'y ait pas de biais de publication, il faut que le nuage de
points représente un entonnoir symétrique (représenté par le triangle
blanc), avec autant d'études de part et d'autres de la valeur moyenne.
Notre graphique nous montre qu'il n'y a pas de biais de publication
(test for funnel plot asymmetry: z = 1,0926, p = 0,2746).

Figure 7 : Graphique en entonnoir montrant la taille d'effet (z) des
études en fonction de leur précision (Erreur type de la taille d'effet).
Les études étudiant la production des enfants (tests : CDI production)
sont représentées par les cercles rouge. Les études étudiant la
compréhension des enfants (tests : CCT, PPVT, CDI compréhension) sont
représentées par les cercles blanc. Les deux cercles avec des croix
représentent nos études en crèches. Celui en rouge correspond à notre
test du CDI pour les données de production, et celui en blanc correspond
à notre test sur tablette tactile (CCT) étudiant la compréhension de
l'enfant.

Une méta-régression révèle que la taille d'effet, quand toutes les
données sont prises en compte est de 0,25. Lors de notre première étude
(tests expérimentaux), nous avons observé un effet plus fort pour le CDI
production (r = 0,493) que pour le test de compréhension sur tablette (r
= 0,337), ce qui va dans le même sens que les résultats de Vogt et ses
collaborateurs (2015), qui observent de plus grands effets du niveau
d'études de la mère sur la production que sur la compréhension au
Mozambique. Nous avons donc vérifié si le mode de test avait un impact
sur le coefficient d'association. Une méta-régression ajustée sur les 13
données de compréhension des enfants est de 0,29, alors que celles pour
les 9 données de production est de 0,19. Contrairement à notre étude et
celle de Vogt, les tailles d'effet pour les études étudiant la
compréhension de l'enfant sont généralement plus grandes que celles
étudiant la production (figure 7), mais cette différence n'est pas
significative.

Figure 8 : Taille de l'effet en fonction de l'âge de l'enfant et du type
de test

Nous avons ensuite observé si la taille d'effet était influencée par
l'âge. D'après la figure 8, nous pouvons voir que la taille d'effet
augmente avec l'âge pour les résultats des mesures de compréhension de
l'enfant, mais cela n'est pas le cas pour les résultats de production,
pour lesquels la taille d'effet ne change pas avec l'âge. Une
méta-régression déclarant le type de mesure (production, compréhension),
l'âge, ainsi que leur intéraction, ne révèle néanmoins pas une
intéraction significative, mais seulement un effet principal de l'âge.
Discussion

References






\end{document}
